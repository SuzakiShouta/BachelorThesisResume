\documentclass{jarticle}
\pagestyle{empty}
\usepackage[dvipdfmx]{graphicx}
\usepackage{booktabs}
\setlength{\topmargin}{-10.4mm}
\setlength{\headheight}{0mm}
\setlength{\headsep}{0mm}
\setlength{\textheight}{262mm}
\setlength{\textwidth}{180mm}
%\setlength{\topskip}{7mm}
\setlength{\evensidemargin}{-10.4mm} 
\setlength{\oddsidemargin}{-10.4mm} 
\setlength{\columnsep}{8mm}
% \setlength{\footskip}{12mm}
\usepackage{subfigure}
\usepackage{setspace}
\usepackage{here}


% 行間調整
\setstretch{0.9}

%sectionのフォントサイズ修正
\makeatletter
\def\section{\@startsection {section}{1}{\z@}{2.5ex plus -1ex minus -.2ex}{1.3 ex plus .1ex}{\large\bf}}
\makeatother 

%subsectionのフォントサイズ修正
\makeatletter
\def\subsection{\@startsection {subsection}{1}{\z@}{1.5ex plus -1ex minus -.4ex}{0.3 ex plus .1ex}{\bf}}
\makeatother 

\usepackage{amsmath} 
\begin{document}
\twocolumn[

\begin{center}
%タイトル
{\LARGE \textbf{時空間フェンシングに基づく\\クラウドセンシングプラットフォームに関する研究
}}\\
%サブタイトル
%{\Large \textbf{必要に応じてサブタイトル}}
\end{center}

\begin{center}
% 著者
\begin{tabular}{cccc}
% 1名の場合
%\multicolumn{4}{c}{K11001 愛工総和}\\
% 2名の場合
%& K11002 愛工七音 & X11003 愛工頼音 &\\
% 3名の場合
%K11001 愛工総和 & K11002 愛工今鹿 & X11003 愛工姫星&\\
% 4名の場合
& K17090 土本涼雅 & K17125 宮川信人 &\\
% 指導教員
\multicolumn{4}{c}{\textbf{指導教員} 梶克彦}
\end{tabular}
\hspace{2zw}
\end{center}
]

%--------------------------------------------
\section{はじめに}

\label{sec:intro}
近年,高機能センサを備えたスマートフォンが増加し,豊富なセンサが利用できるようになっている.
そのスマートフォンのセンシング能力を活かす試みとして,クラウドセンシングがある.
クラウドセンシングは幅広いデータ収集かつセンシングコストを削減できるため,研究や調査などで採用されている.
研究として,雑踏音や歩行動作をセンシングして混雑状況をリアルタイムで推定する研究\cite{ura}や,取得した音圧データを基に騒音レベルを共有する研究\cite{iwa}など多くある.
クラウドセンシングを利用するためには専用システムの開発が必要だが,イニシャルコストが大きくかかる.
また,多くのユーザにセンシングの協力を促すためには,物理的及び心理的コストの削減などの課題がある.
物理的コストとしては協力にかかる操作,通信といった負担が挙げられる.
心理的コストとしては協力者のプライバシ障壁によるデータ提供への心配やセンシングに対する不安などが挙げられる.
また,クラウドセンシングでは協力者の個人情報を多く取り扱うため,セキュリティやプライバシ保護の対策が必須である.

本研究ではクラウドセンシングの簡易的な利用と多様なデータ収集を行い,研究や調査におけるイニシャル及びランニングコストの大幅な軽減を目的とする.
また,協力者の発生し得る物理的及び心理的コストの軽減し,ユーザのセンシングの協力かつ継続を促進する.
我々はこの目的を実現するために,時間とエリアを制限する時空間フェンシングの概念を提案し,それに基づいたクラウドセンシングプラットフォーム「$\text{Lav.}^{+}$(ラヴラス)」を構築する.

% \section{関連研究}
既存研究として,ゲーミフィケーションを用いて協力者のセンシングへの協力意欲を向上させ,モチベーションを保つクラウドセンシングプラットフォーム\cite{tex}がある.
我々の研究は,基本的にスマートフォンの操作はほとんどなく,自動でセンシングを行う.
そのため,意識的にセンシングに協力する既存研究とは違って,無意識的に協力が可能となり,センシングに対する負担が少なく協力継続できると考える.
また,無意識的であるため,普段の人々の振る舞いや行動をセンシングしたい場合には,ラヴラスは適している.


%--------------------------------------------
\section{$\text{Lav.}^{+}$(ラヴラス)}

本研究では,まず時間とエリアを制限する時空間フェンシングの定義を行い,それに基づいたクラウドセンシングプラットフォーム「$\text{Lav.}^{+}$(ラヴラス)」を構築する.
ラヴラスは,センシングデータや各センシングプロジェクトを管理するサーバ(以下,サーバ),依頼者用のセンシングプロジェクト管理Webアプリケーション(以下,Webアプリ), 協力者用のセンシングスマートフォンアプリケーション(以下,スマホアプリ)によって構成される.

\subsection{時空間フェンシングの定義}

時空間フェンシングは「ジオフェンシングに時間要素を追加し拡張したフェンシング手法」として定義する.
ジオフェンシングとはGPSやWi-Fiなどで仮想的な境界を生成し,その境界に侵入した,あるいは出たときに特定のサービスを行う仕組みである.
時空間フェンシングによって時間とエリアで境界を区切ると,依頼者は様々なシチュエーションを指定したクラウドセンシングが可能となる.
一方で,時空間フェンシングはエリア内にいる間,常時センシングや車での移動など時間やエリアに依存しないデータ収集には適さない.
協力者のクラウドセンシングに対するプライバシ障壁は,時空間フェンシングによる時間とエリアの制限で軽減できる.


% 図を参照するときの番号を自動で入れるには \ref{ラベル名} と書く
% もし,下の図を参照するなら \ref{fig:1} となる.
% 例: システム概要図を図\ref{fig:1}に示す.

\subsection{ラヴラスの流れ}

\begin{figure}[H]
    \centering
    \includegraphics[width=80mm]{fig1.png}
    \caption{ラヴラスの流れ}
    \label{fig:1}
\end{figure}
% \vspace{-2zh}

ラヴラスの流れを図1に示す.
ラヴラスの流れはセンシングプロジェクトの定義,時空間フェンシング,センシング依頼通知,センシング,アップロードの順で行う.
依頼者は専用のWebアプリにて,センシングしたいエリアや時間帯,センシングの目的や概要などをセンシングプロジェクトとして細かく定義する.
協力者には事前にラヴラス専用のスマホアプリをインストールしてもらう.
専用のスマホアプリではセンシングプロジェクトに応じて時空間フェンシングを行い,協力者が特定の時間帯かつ特定のエリアにいる場合のみヘッドアップ通知が送られる.
協力者は通知のタップで本スマホアプリの通知画面を開き,依頼されているセンシングプロジェクトの内容を確認し,承諾か拒否かの判断を行う
センシングは常にバックグラウンドで行い,センシング終了後は自動でサーバ側にアップロードされる.
依頼者は本Webアプリにて,アップロードされたセンシングデータをダウンロード可能である.

\subsection{ラヴラスの実装}

サーバはWebアプリ及びスマホアプリのどちらにも親和性が高いJSONベースのREST API を設計した
データベースで管理するデータは,匿名の依頼者によるクラウドセンシングは行わないための依頼者情報,センシングプロジェクト,使用センサの種類やセンシングを行う時空間などの設定,収集されたセンシングデータである.
サーバによって管理する各データベースは,APIのエントリポイントとしてアクセスが可能となっている.
また,サーバではセンシティブな情報を取り扱うため,セキュリティ対策としてJWTトークン認証によってアクセスを制御する.
もし,権限のない第三者がセンシングデータのダウンロード等のアクセスを行おうとしてもアクセスはできない.

次に,依頼者専用のWebアプリとしてAPIサーバとの親和性が高いSingle Page Applicationを採用し実装を行った.
依頼者のユーザ登録やセンシングプロジェクトの作成,各センシングプロジェクトを基に収集されたセンシングデータのダウンロード等が可能である.
依頼者は,まずユーザ登録を行い,センシングプロジェクトを作成する.
作成した各センシングプロジェクトはサーバによって管理され,スマホアプリが適時それを取得する.
そして,Webアプリの管理画面より提供されたセンシングデータのダウンロードやメタ情報の確認ができる.
メタ情報とは,センシングを行った協力者の端末やOSのバージョン等の情報が記載されたものである.

\begin{figure}[H]
    \centering
    \includegraphics[width=60mm]{fig2.png}
    \caption{スマホアプリの通知}
    \label{fig:2}
\end{figure}

協力者がセンシングプロジェクト毎にスマホアプリをインストールする手間を省くため,協力者専用のAndroidアプリケーションを作成した.
本スマホアプリはサーバと連携し,各センシングプロジェクトの受信,メタ情報とセンシングデータの送信を行う.
時空間フェンシングは位置情報取得を最小限にするために,まず時間判定を行い,センシングプロジェクトで指定された開始時間にエリア判定を行う.
位置情報はGPSを用いて取得する.
協力者が指定センシングエリア内に進入したか否かは,どんな複雑な多角形のエリアでも対応できるよう点の内部判定で判断する.
協力者が指定された時空間内に入ったら,より気がつきやすい振動ありのヘッドアップ通知を送る(図2左).
依頼通知画面では現在地と指定エリア,時間帯,センシングプロジェクトや依頼者情報の内容を表示し,協力者の確認により自身のセンシングに対する不安や疑念を解消させる(図2右).
センシング終了後,協力者の通信の負担を軽減するため,協力者のスマートフォンがWi-Fiに接続中である場合にのみ,センシングデータとメタ情報は自動でサーバへ送信する.
また,協力者の操作を最小限にするため,時空間フェンシングやセンシング,センシングデータのアップロードなどアプリの大部分はバックグラウンドで行う.



%--------------------------------------------
\section{動作検証}
依頼者側のセンシングプロジェクトの作成やセンシングデータのダウンロード,協力者側の時空間フェンシングや通知,センシングプロジェクトに基づいたセンシング,センシングデータの送信などラヴラスの主な流れが正常に動作するか検証する.
今回の動作検証は,13時から14時までの愛知工業大学4号館別館での活動量を加速度センサで調査するといったシナリオに基づいて行う.
結果として,我々が期待した通りの動作が確認できた.
まずWebアプリでシナリオを基にしたセンシングプロジェクトの作成,スマホアプリでセンシングプロジェクトの受信,センシングデータの送信により,Webアプリ・スマホアプリ双方のサーバとの連携が確認できた.
時空間フェンシングは,エリアが屋内なため,GPSでエリア判定を行うと位置情報誤差はあったが,動作は確認できた.屋内でのエリア判定は今後の課題とする.
指定した時空間内に進入すると,ヘッドアップ通知が表示され,それをタップし,依頼を承認するとセンシングが開始され,正常な動作が確認できた.
センシング後にサーバに送信されたセンシングデータを確認したところ,シナリオ通りに加速度が収集されていたため,動作検証は成功とする.

%--------------------------------------------
\section{今後の課題}
\label{sec:reference}
% ラヴラスの課題としてセンシングデータの提供方法が挙げられる.
% 現状での提供方法としては,ダウンロードという形を取っているが,協力者のプライバシ保護のため,協力者の要請に応じて提供済みのセンシングデータを削除する必要がある.
% 依頼者がセンシングデータのダウンロードを行ってしまった場合,サーバ上のセンシングデータの削除は可能であるが,ダウンロード済みのデータの削除は不可能である.
% こういった状況を回避するためにも,センシングデータの提供はWebアプリ上での閲覧のみにするといった対策が必要である.
% %また,現状の時空間フェンシングのエリア判定はGeoJSONによる定義のみである.
% %これは,GPS情報の不安定な屋内やGPSによる判定が困難な狭い空間などでは適用が困難である.
% %そのため,エリア判定にはWi-FiやBLEビーコンを用いた電波による位置測定や屋内における鉄骨などから発せられる磁気を用いた地磁気測位などを使用しより詳細なエリアを定義できるようにする必要がある.
% また,ラヴラスはまだ運用や評価には至っていない.
% そのため,現状挙げられた課題を解決するとともに,実際にクラウドセンシングを行い評価実験を実施した上でプラットフォームとして運用できるシステムを構築する必要がある.
ラヴラスの課題として利用規約の制定が挙げられる.
例えば,協力者のプライバシ保護のため,協力者の要請に応じてサーバ上の提供済みセンシングデータを削除する必要がある.
しかし,ラヴラスのセンシングデータの提供方法はダウンロードという形を取っているため,利用規約として一定期間以上のセンシングデータの保持を禁止するといった規約を制定する.
% また,実装面において現在採用してるセンシングデータのフォーマットはファイルサイズが重量化する傾向があるためその軽量化対策や時空間フェンシングにおけるエリア判定は現在GPSのみで行っているためGPS情報の不安定な屋内やGPSによる判定が困難な狭い空間でもエリア判定が可能な電波や磁気を用いたエリアを定義可能にする必要がある.
また,ラヴラスはまだ運用や評価には至っていない.そのため,現状挙げられた課題を解決するとともに,実際にクラウドセンシングを行い評価実験を実施した上でプラットフォームとして運用できるシステムを構築する必要がある.
%--------------------------------------------

% \bibliography{reference}
% \bibliographystyle{junsrt}

\begin{thebibliography}{9}
    \bibitem{ura}西村 他.スマートフォンを活用した屋内環境における混雑センシング,情報処理学会論文誌,Vol.55,No.12,pp.2511-2523,(2014).
    \bibitem{iwa}岩井 他.コミュニティによるスマートフォンを利用した騒音センシングシステムの構築,情報科学技術フォーラム講演論文集,Vol.10,No.4,pp.293-294,(2011).
    \bibitem{tex}松田 他. 多様なユースケースに対応可能なユーザ参加型モバイルセンシング基盤の実装と評価,マルチメディア,分散,協調とモバイル(DICOMO)シンポジウム論文集,Vol.2016,pp.1042-1050,(2016).
\end{thebibliography}

\end{document}

%%% Local Variables: 
%%% mode: japanese-latex
%%% TeX-master: t
%%% End: 
