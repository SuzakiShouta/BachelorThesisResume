\documentclass{jarticle}
\pagestyle{empty}
\usepackage[dvipdfmx]{graphicx}
\usepackage{booktabs}
\setlength{\topmargin}{-10.4mm}
\setlength{\headheight}{0mm}
\setlength{\headsep}{0mm}
\setlength{\textheight}{262mm}
\setlength{\textwidth}{180mm}
%\setlength{\topskip}{7mm}
\setlength{\evensidemargin}{-10.4mm} 
\setlength{\oddsidemargin}{-10.4mm} 
\setlength{\columnsep}{8mm}
% \setlength{\footskip}{12mm}
\usepackage{subfigure}
\usepackage{setspace}
\usepackage{here}


% 行間調整
\setstretch{0.9}

%sectionのフォントサイズ修正
\makeatletter
\def\section{\@startsection {section}{1}{\z@}{2.5ex plus -1ex minus -.2ex}{1.3 ex plus .1ex}{\large\bf}}
\makeatother 

%subsectionのフォントサイズ修正
\makeatletter
\def\subsection{\@startsection {subsection}{1}{\z@}{1.5ex plus -1ex minus -.4ex}{0.3 ex plus .1ex}{\bf}}
\makeatother 

\usepackage{amsmath} 
\begin{document}
\twocolumn[

\begin{center}
%タイトル
{\LARGE \textbf{特定の時空間への進入時に\\自動センシングするアプリケーションに関する研究
}}\\
%サブタイトル
%{\Large \textbf{必要に応じてサブタイトル}}
\end{center}

\begin{center}
% 著者
\begin{tabular}{cccc}
% 1名の場合
%\multicolumn{4}{c}{K11001 愛工総和}\\
% 2名の場合
%& K11002 愛工七音 & X11003 愛工頼音 &\\
% 3名の場合
%K11001 愛工総和 & K11002 愛工今鹿 & X11003 愛工姫星&\\
% 4名の場合
& K18054 & 須崎翔太\\
% 指導教員
\multicolumn{4}{c}{\textbf{指導教員} 梶克彦}
\end{tabular}
\hspace{2zw}
\end{center}
]

%--------------------------------------------
\section{はじめに}

近年,高性能センサを備えたスマートフォンが増加し,豊富なセンサが利用できるようになっている.
そのスマートフォンのセンシング能力を活かす試みとして,クラウドセンシングがある.
クラウドセンシングは幅広いデータを収集でき,センシングコストを削減できるため,研究や調査などで採用されている.
研究として,雑踏音や歩行動作をセンシングして混雑状況をリアルタイムで推定する研究\cite{ura}や,取得した音圧データを基に騒音レベルを共有する研究\cite{iwa}など多くある.

クラウドセンシングにはいくつかの課題がある,
クラウドセンシングをの課題として,専用システムの開発コスト,センシングによるプライバシの侵害,適切なセンサデータの確保,協力者のモチベーション維持などが挙げられる.
我々は,この課題の内,専用システムの開発コスト,センシングによるプライバシの侵害,協力者のモチベーション維持を解決する為に,時空間フェンシングに基づくクラウドセンシングプラットフォーム「ラヴラス」を提案した.
クラウドセンシングプラットフォームを構築し,クラウドセンシングの容易利用と多様なデータ収集ができるようにして,研究や調査におけるコスト(時間・費用・手間)を大幅に減少する.
さらに,クラウドセンシングプラットフォームの設計基盤として時空間フェンシングを提案し,データを収集したい人(以下,依頼者)はセンシングする範囲を定義しやすく,センシングに協力する人(以下,協力者)はセンシングされている時を明確に認識できる.
ラヴラスはwebアプリ,サーバ,モバイルアプリで構成されており,依頼者はwebアプリ,協力者はモバイルアプリを使用する.
本研究はラヴラスのモバイルアプリケーションに関する研究である.

クラウドセンシングに必要なセンサを搭載したセンシング端末にはいくつかの課題がある.
クラウドセンシングに専用のデータロガーを使用した場合,データロガーの確保や配布,回収するコストがかかる.
クラウドセンシングにモバイルアプリケーションを使用した場合の課題として,協力者の物理的コストと心理的コストが挙げられる.
協力者の物理的コストとして,協力者の負担と,端末の負担が挙げられる.
協力者がクラウドセンシンングに協力するまでの操作が多いと,協力者がクラウドセンシングに協力しなくなる.
また,アプリの端末のデータ通信量が多い,バッテリーの消耗が激しい,センサログデータのデータサイズが大きく端末の容量を圧迫するなど,協力者の端末に負担が多い場合も同様である.
協力者への心理的コストとして,第三者へのセンシングデータ提供に対する不安や個人情報悪用の心配などのプライバシ意識が挙げられる.
本研究が対象とする課題は,協力者の負担と,アプリ端末のデータ通信量及び心理的コストである.

本研究では協力者のディスインセンティブ要素の減少を目的とし,ユーザのセンシングの協力かつ継続を促進する.
そのために協力者の発生しうる物理的及び心理的コストの減少を行う.
物理的コストの減少として,協力者の操作を最小限にし,センシングデータアップロードはWi-Fi下で行う.
心理的コストの減少として,協力者がクラウドセンシングの内容に納得し,協力すると判断した場合のみセンシングを行う.
また,協力者のプライバシーの侵害を防ぐために,本アプリでアップロードされるセンサデータ等はすべて匿名化及び抽象化され,協力者は自身のセンサデータの削除及び削除申請ができる.

既存研究として,ゲーミフィケーションを用いて協力者のセンシングへの協力意欲を向上させ,モチベーションを保つクラウドセンシングプラットフォーム\cite{tex}がある.
我々の研究は,協力者のディスインセンティブ要素を減少し,利用継続を促す.そのため,依頼者はインセンティブコストが発生せず,協力者は安心してクラウドセンシングに参加できる.


\section{時空間フェンシングに基づいたクラウドセンシングプラットフォーム}
本章ではまず時空間フェンシングの概念を定義し,次に時空間フェンシングに基づくクラウドセンシングプラットフォームの全体図について述べる.

\subsection{時空間フェンシングの定義}
時空間フェンシングは「ジオフェンシングに時間要素を追加し拡張したフェンシング手法」として定義する.
時空間フェンシングのメリットとして,センシングする範囲が認識しやすい点が挙げられる.
依頼者はクラウドセンシングの範囲を定義しやすく,協力者はセンシングされる範囲を適切に認識した上でセンシング依頼に承諾,拒否できる.
時空間フェンシングのデメリットとして,時間と空間に依存しないクラウドセンシングに適さない点である.
例えば,空間に依存しない移動する電車内でのクラウドセンシングや,時間に依存しない雨が降った時のみのクラウドセンシングには適さない.

\subsection{時空間フェンシングに基づくクラウドセンシングプラットフォーム}

\begin{figure}[H]
    \centering
    \includegraphics[width=80mm]{Lavlus.png}
    \caption{ラヴラスの全体図}
    \label{fig:1}
\end{figure}
\vspace{-2zh}

ラヴラスの一連の流れは「Webアプリでセンシングプロジェクトの定義」,「時空間フェンシング」,「センシング依頼の承諾」,「自動的にセンシング」,「Wi-Fi環境下で自動的にアップロード」,「データ利用」の順で行う.
依頼者はプロジェクト管理Webアプリにて,センシング依頼の内容を細かく定義し,センシングプロジェクトを作成する.
スマホアプリ側ではセンシングプロジェクトに応じて,3.1節の定義をもとに時空間フェンシングを行い,協力者が指定された時間帯かつエリアにいる場合のみ通知が送られる.
協力者がセンシング依頼に承諾するとセンシングして,その後センサデータをアップロードする.

本プラットフォームは時空間を適切に設定でき,無意識化でセンシングするクラウドセンシングのみ使用できる.
例えば遊園地の経営企業が遊園地の入場者の動向を知るために移動履歴をセンシングしようとしたと仮定する.
その場合,時間は遊園地の開園時間から閉園時間,空間は遊園地内,必要なセンサデータは位置情報,加速度センサと設定できる.
% (もう一つ例を書きたい,)


\section{特定の時空間への進入時に自動センシングするアプリケーション}
\subsection{ラヴラスのモバイルアプリケーションの要求仕様}
ラヴラスのモバイルアプリケーションはセンシングプロジェクトダウンロード,時空間フェンシング,センシング依頼の承諾,自動的にセンシング,Wi-Fi環境下で自動的にアップロードの順で行う.
センシングプロジェクトをダウンロードする時,端末の通信量,データ容量を圧迫しない為に協力者が参加する可能性があるセンシングプロジェクトのみダウンロードする必要がある.
また,協力者の物理的コストを減少させるために,協力者へ通知と協力者自身の操作を最小限に抑えたり,端末のデータ通信量を圧迫しない為にセンシングが終わった後,Wi-Fiに接続している時に自動でアップロードされる必要がある.
% 依頼者の制作したセンシングプロジェクトに参加する可能性が高い(時空間に近い,進入している.)協力者のみにセンシング依頼通知を送る.
% また,一度センシング依頼に承諾,拒否した場合そのセンシングプロジェクトから通知は発行されない.また,時空間に進入している間,バックグラウンドで自動でセンシングされる.
協力者の心理的コストを減少するために,協力者はすでにセンシングに承諾したセンシングプロジェクトにもセンシング拒否ができたり,センサデータの削除,既にアップロードした場合は削除申請が出せる必要がある.
% また,送信したセンサデータに削除申請ができる.

様々なクラウドセンシングに対応する為,多くのセンサに対応する必要がある.
また,アップロードされるセンサデータは協力者のプライバシを侵害しない為に,匿名化及び抽象化する必要がある.

\subsection{時空間への進入時に自動センシングするアプリケーションの実装}
依頼者の制作したセンシングプロジェクトに対応したセンシングをするスマートフォンアプリケーションとしてAndroidアプリを作成した.
本アプリは4.1章で述べた内,時空間フェンシング,センシング依頼の承諾,自動的にセンシングのみ実装した.

\begin{figure}[H]
    \centering
    \includegraphics[width=80mm]{myApp.png}
    \caption{時空間への進入時に自動センシングするアプリケーション}
    \label{fig:2}
\end{figure}
\vspace{-2zh}

\subsubsection{時空間フェンシングの実装}
ジオフェンスには緯度経度,Wi-Fi,BLEビーコンなどが使われているが,本アプリでは依頼者と協力者が認識しやすい緯度経度を使用した.
また,ジオフェンシングが複雑な矩形である場合に対応する為,ジオフェンシングの内外判定にポリゴンの内外判定アルゴリズムを使用した.

また,確実に時空間に進入した場合のみセンシングする,時空間に進入する可能性が高い協力者にセンシング依頼通知を発行するなど,様々なシチュエーションに対応するため,時空間を拡大,縮小できるマージンを実装した.
しかし,ジオフェンシングが複雑な矩形である場合に対応する為,協力者の位置情報にマージンを設けた.

本アプリは協力者が時空間に進入しているかの判定のため,一定間毎に位置情報と現在時刻を取得する.
この時,マージンを含め,時空間ともに進入していた場合,時空間へ進入したと判定する.
% 複雑な矩形のマージンに対応できるように端末にマージンを設けた. マージンはここで話す

\subsubsection{センシング依頼通知の実装}
協力者へのセンシング依頼通知を減らすために,センシング依頼通知はセンシングプロジェクトに参加する可能性が高い場合のみ通知を発行する.
その為,時空間を広げるようにマージンを取り,その時空間へ進入した場合センシングプロジェクトに参加する可能性が高いを判断し,通知を発行する.

協力者がセンシング依頼通知をタップすると,センシング依頼画面が立ち上がり,依頼者の名前や使用するセンサ,時空間が提示される.
提示された情報に協力者が納得した場合,センシング承諾ボタンを押すだけでクラウドセンシングに協力できる.

\subsubsection{自動センシングの実装}
% 入った時と出た時でマージンが違う.
% どう言った時でセンシングが始まるべきか,おわるべきか.
協力者が時空間に進入し,センシング依頼に承諾している場合,バックグラウンドで自動でセンシングされる.
確実に協力者の時空間への進入,退出を判定するため,進入時は時空間を狭くするマージン,退出時は時空間を広くするマージンを取る.

クラウドセンシングプラットフォームとして多くのセンサと自由な周波数に対応し,プライバシを侵害するセンサデータがセンシングされないようにした.
その為,プライバシを侵害する可能性が高いセンサデータは抽象化されている.

\section{動作検証}
本研究の動作検証は特定の時空間に進入時のみセンシングできているか,プラットフォームとして複数のユースケースを想定して適切にセンシングできているかの2つを行う.
\subsection{時空間フェンシングの動作検証}
\subsection{ユースケースを想定した動作検証}
天候によって所要時間が変化する地図アプリを作成したい人がいたとする.

研究室の管理者が,研究室内でどれだけコミュニケーションが取れているか測定しようとしたとする.


\section{おわりに}
\subsection{まとめ}
研究目的
動作検証では時空間フェンシングが適切に行えているか,実際のユースケースを想定して適切にセンシングできているか検証した.結果何がわかった.
\subsection{今後の課題}
今後の課題として,今回実装に至らなかった時空間フェンシングに基づくクラウドセンシングプラットフォームにおけるモバイルアプリケーションに必要な機能の実装が挙げられる.

時空間フェンシングにGPSを使用しているのでGPSの精度が落ちる屋内で動作が不安定になる点.

% \bibliography{reference}
% \bibliographystyle{junsrt}

\begin{thebibliography}{9}
    \bibitem[1]{ura}西村 他.スマートフォンを活用した屋内環境における混雑センシング,情報処理学会論文誌,Vol.55,No.12,pp.2511-2523,(2014).
    \bibitem[2]{iwa}岩井 他.コミュニティによるスマートフォンを利用した騒音センシングシステムの構築,情報科学技術フォーラム講演論文集,Vol.10,No.4,pp.293-294,(2011).
    \bibitem[3]{tex}松田 他. 多様なユースケースに対応可能なユーザ参加型モバイルセンシング基盤の実装と評価,マルチメディア,分散,協調とモバイル(DICOMO)シンポジウム論文集,Vol.2016,pp.1042-1050,(2016).
\end{thebibliography}

\end{document}


% 図を参照するときの番号を自動で入れるには \ref{ラベル名} と書く
% もし,下の図を参照するなら \ref{fig:1} となる.
% 例: システム概要図を図\ref{fig:1}に示す.

% \begin{figure}[H]
%     \centering
%     \includegraphics[width=80mm]{プレゼンデーション1.png}
%     \caption{本研究の目的}
%     \label{fig:1}
% \end{figure}
% \vspace{-2zh}


%%% Local Variables: 
%%% mode: japanese-latex
%%% TeX-master: t
%%% End: 
