\documentclass{jarticle}
\pagestyle{empty}
\usepackage[dvipdfmx]{graphicx}
\usepackage{booktabs}
\setlength{\topmargin}{-10.4mm}
\setlength{\headheight}{0mm}
\setlength{\headsep}{0mm}
\setlength{\textheight}{262mm}
\setlength{\textwidth}{180mm}
%\setlength{\topskip}{7mm}
\setlength{\evensidemargin}{-10.4mm} 
\setlength{\oddsidemargin}{-10.4mm} 
\setlength{\columnsep}{8mm}
% \setlength{\footskip}{12mm}
\usepackage{subfigure}
\usepackage{setspace}
\usepackage{here}


% 行間調整
\setstretch{0.9}

%sectionのフォントサイズ修正
\makeatletter
\def\section{\@startsection {section}{1}{\z@}{2.5ex plus -1ex minus -.2ex}{1.3 ex plus .1ex}{\large\bf}}
\makeatother 

%subsectionのフォントサイズ修正
\makeatletter
\def\subsection{\@startsection {subsection}{1}{\z@}{1.5ex plus -1ex minus -.4ex}{0.3 ex plus .1ex}{\bf}}
\makeatother 

\usepackage{amsmath} 
\begin{document}
\twocolumn[

\begin{center}
%タイトル
{\LARGE \textbf{特定の時空間への進入時に\\自動センシングするアプリケーションに関する研究
}}\\
%サブタイトル
%{\Large \textbf{必要に応じてサブタイトル}}
\end{center}

\begin{center}
% 著者
\begin{tabular}{cccc}
% 1名の場合
%\multicolumn{4}{c}{K11001 愛工総和}\\
% 2名の場合
%& K11002 愛工七音 & X11003 愛工頼音 &\\
% 3名の場合
%K11001 愛工総和 & K11002 愛工今鹿 & X11003 愛工姫星&\\
% 4名の場合
& K18054 & 須崎翔太\\
% 指導教員
\multicolumn{4}{c}{\textbf{指導教員} 梶克彦}
\end{tabular}
\hspace{2zw}
\end{center}
]

%--------------------------------------------
\section{はじめに}

\subsection{研究背景}
\label{sec:intro}
近年,高機能センサを備えたスマートフォンが増加している.

クラウドセンシングは幅広いデータ収集かつセンシングコストを削減できるため,様々な研究で採用されている.

しかし,クラウドセンシングにはいくつかの課題があり,それを解決するために,我々は時空間フェンシングに基づくクラウドセンシングプラットフォーム「ラヴラス」を提案した.
% プラットフォームにして,クラウドセンシングの容易利用と多様なデータ収集ができるようにして,研究や調査におけるコスト(時間・費用・手間)を大幅に軽減する.
% 時空間フェンシングを提案して依頼者はセンシングする範囲を定義しやすく,協力者はセンシングされている時を明確に.

本研究はラヴラスのモバイルアプリケーションに関する研究である.
% クラウドセンシングプラットフォームを作ろうとしていて,自分はモバイルアプリ開発をしている.
% データ収集をしたい人(依頼者)はwebアプリから,センシングに協力する人(協力者)はスマホアプリから
% 依頼者が本プラットフォームを利用してセンシング依頼(センシングプロジェクト)を作成する.

\subsection{クラウドセンシングの課題}
依頼者側の課題として,サーバやアプリなどの専用システムの開発にかかるイニシャルコストやランニングコストが挙げられる.

また,依頼者の知識不足により,本来扱ってはいけない協力者のプライバシを侵害するセンサデータを集めてしまったり,協力者にセンシングがプライバシを侵害する危険性を説明しきれない可能性がある.

クラウドセンシングで協力者から集めたデータのクオリティが依頼者の要求するレベルに達しない場合がある.

協力者側の課題として,センサデータの提供にはディスインセンティブ要素が多い点が挙げられる.
% クラウドセンシングにはセンサデータを提供してくれる協力者が必須.
% 例えば協力者はいつセンシングされているかわからない.どのようにセンシングされているかわからない.
% プライバシを侵害されるような危険なセンシングをされている可能性がある.

本研究が対象とする課題はクラウドセンシングで重要である,依頼者側のサーバやアプリなどの専用システムの開発にかかるイニシャルコストやランニングコスト,依頼者の知識不足により発生する,危険なセンサデータの収集,協力者側のセンサデータ提供に対するディスインセンティブ要素である.


\subsection{センシング端末の課題}
クラウドセンシングに必要なセンサを搭載したセンシング端末にはいくつかの課題がある.

クラウドセンシングに専用のデータロガーを使用した場合の課題として,物理的コストが挙げられる.

% データロガーの確保や配布,回収
クラウドセンシングにモバイルアプリケーションを使用した場合の課題として,協力者の物理的コストと心理的コストが挙げられる.

協力者は複数のクラウドセンシングに協力すると協力した分だけ専用のアプリケーションをインストールしなくてはならない.

協力者のアプリケーション内での操作やアプリケーションを使用する際のデータ通信量などの負担が多いと,協力者はアプリケーションを放置または削除してしまう.

第三者へのセンシングデータ提供に対する不安や個人情報悪用の心配などのプライバシ意識により協力者獲得は容易ではない.
% 協力者が提供したくないデータは送信しない,送信済みの場合は削除申請ができなければならない.


\subsection{研究目的}
本研究では協力者のディスインセンティブ要素の軽減を目的とし,ユーザのセンシングの協力かつ継続を促進する.

そのために協力者の発生しうる物理的及び心理的コストの軽減を行う.
% 協力者の物理的コスト面の課題(依頼者側のサーバやアプリなどの専用システムの開発にかかるイニシャルコストやランニングコスト):個別アプリケーションのインストールをなくして,一つにする.
% また,スマートフォンの操作,通知を最小限にする.時空間に進入しないなら通知を出さない.一度承諾,拒否したらもう通知を出さない.

センシングデータアップロードはWi-Fi下で行う.

心理的コスト面の課題:センシングデータ提供に対する不安は,依頼者の情報を提示して,センシング依頼に承諾してもらう.不安があればセンシング依頼に承諾した後でも拒否できる.未送信のセンシングデータは削除ができ,送信済みなら削除申請が出せる.

また,協力者のプライバシーの侵害を防ぐために,本アプリでアップロードされるセンサデータ等はすべて匿名化及び抽象化する.


\subsection{論文構成}


\section{関連研究}
\subsection{クラウドセンシングに関する研究}
幅広いデータ収集かつセンシングコストを削減できるクラウドセンシングを利用している研究はいくつかある.

これらの研究ではクラウドセンシングシステムの開発などには大きなコストがかかると考えられる.

\subsection{クラウドセンシングプラットフォームに関する研究}
実際に運用を行っているクラウドセンシングプラットフォームとして,OhmageやAWAREなどがある.

クラウドセンシングは協力者の確保が非常に重要であるため,様々な方法でモチベーションを向上・維持させる必要がある.

本研究ではディスインセンティブ要素を軽減する.

\subsection{センシング端末に関する研究}
クラウドセンシングのセンシング端末として様々な端末が使用されている.

例えば,スマートフォンが使用されている.

スマートフォンのクラウドセンシングは協力者がそのクラウドセンシングに対応したアプリケーションをそれぞれインストールする必要がある.

スマートフォンを使用しない例として,市販の環境センサや,専用に開発されたものがある.
% オムロン(IoTセンシングによる賃貸物件快適度推定のためのデータ収集)
% アメダス,日本海溝海底地震津波観測網

スマートフォンを使用せず,市販の環境センサや,専用に開発したものは,長時間のセンシングや大規模なセンシングが可能であるが,イニシャルコストとランニングコストがかかる.


\section{時空間フェンシングに基づいたクラウドセンシングプラットフォーム}
本章ではまず時空間フェンシングの概念を定義し,次に時空間フェンシングに基づくクラウドセンシングプラットフォームの全体図について述べる.

本クラウドセンシングプラットフォーム「Lavlus」(以下,ラヴラス) の命名は,”a view of Laplace’sdemon”「ラプラスの魔の視界」から来ている.

\subsection{時空間フェンシングの定義}
時空間フェンシングは「ジオフェンシングに時間要素を追加し拡張したフェンシング手法」として定義する.

時空間フェンシングのメリットとして,時間とエリアで境界を区切ると依頼者は様々なシチュエーションを指定したクラウドセンシングが可能となる.
% (依頼者はクラウドセンシングの範囲を定義しやすい,依頼者も協力者もセンシングする場所を認識しやすい.)←これもメリット

協力者のクラウドセンシングに対するプライバシ障壁は,時空間フェンシングによる時間と空間の制限で軽減できる
% プライバシ障壁が軽減できるわけではなく,判断しやすい.センシング依頼承諾と合わせていい感じになる.

時空間フェンシングのデメリットとして,時間と空間に依存しないクラウドセンシングに適さない点である.
% 例えば,空間に依存しない移動する電車内でのクラウドセンシングや,時間に依存しない雨が降った時のみのクラウドセンシングなど.

\subsection{時空間フェンシングに基づくクラウドセンシングプラットフォーム}
ラヴラスの一連の流れは「Webアプリでセンシングプロジェクトの定義」,「時空間フェンシング」,「センシング依頼の承諾」,「自動的にセンシング」,「Wi-Fi環境下で自動的にアップロード」,「データ利用」の順で行う.

依頼者はプロジェクト管理Webアプリにて,センシング依頼の内容を細かく定義し,センシングプロジェクトを作成する.

スマホアプリ側ではセンシングプロジェクトに応じて,3.1節の定義をもとに時空間フェンシングを行い,協力者が指定された時間帯かつエリアにいる場合のみ通知が送られる.
% センシング依頼に承諾するとセンシングして,その後アップロードする.(システムから見た時の話なのでマージン,Wi-Fi下の話はしない.)

本プラットフォームは時空間を適切に設定でき,無意識化でセンシングするクラウドセンシングのみ使用できる.

例えば遊園地の経営企業が遊園地の入場者の動向を知るために移動履歴をセンシングしようとしたとする.
その場合,時間は遊園地の開園時間から閉園時間,空間は遊園地内,必要なセンサデータは位置情報と設定できる.
% (もう一つ例を書きたい,)


\section{特定の時空間への進入時に自動センシングするアプリケーション}
\subsection{ラヴラスのモバイルアプリケーションの要求仕様}
ラヴラスのモバイルアプリケーションはセンシングプロジェクトダウンロード,時空間フェンシング,センシング依頼の承諾,自動的にセンシング,Wi-Fi環境下で自動的にアップロードの順で行う.

協力者が本アプリを起動,もしくは起動してから一定時間毎にサーバからセンシングプロジェクトをダウンロードする.
% あいまいな位置情報

協力者が時空間に進入した場合,通知が発行され,センシング依頼画面が立ち上がる.センシング依頼に承諾した場合,時空間に進入している間,バックグラウンドで自動でセンシングされる.

センシングが終わった後,Wi-Fiに接続している時に自動でアップロードされる.

協力者はすでにセンシングに承諾したセンシングプロジェクトにもセンシング拒否ができる.
% また,送信したセンサデータに削除申請ができる.

クラウドセンシングプラットフォームとして,多くのセンサに対応する必要がある.

\subsection{時空間への進入時に自動センシングするアプリケーションの実装}
依頼者の制作したセンシングプロジェクトに対応したセンシングをするためにAndroidアプリを作成した.

本アプリは4.1章で述べた内,時空間フェンシング,センシング依頼の承諾,自動的にセンシングのみ実装した.

\subsubsection{時空間フェンシングの実装}
時空間に進入しているかの判定のため,一定間毎に位置情報を現在時刻を取得する.

複雑な矩形に対応するためにポリゴンの内外判定アルゴリズムを使用する.

ジオフェンシングの境界付近かつ,位置情報が不安定になると進入,退出の判定を繰り返してしまう.これを防ぐためにマージンを設けた.
% 複雑な矩形のマージンに対応できるように端末にマージンを設けた.

\subsubsection{センシング依頼通知の実装}
協力者が時空間に進入するとセンシング依頼の通知が発行される.

協力者が安心してセンシングに協力できるように

\subsubsection{自動センシングの実装}
協力者が時空間に進入し,センシング依頼に承諾している場合,バックグラウンドで自動でセンシングされる.

クラウドセンシングプラットフォームとして多くのセンサと自由な周波数に対応した.

\section{動作検証}
本研究の動作検証は特定の時空間に進入時のみセンシングできているか,プラットフォームとして複数のユースケースを想定して適切にセンシングできているかの2つを行う.
\subsection{時空間フェンシングの動作検証}
\subsection{ユースケースを想定した動作検証}
天候によって所要時間が変化する地図アプリを作成したい人がいたとする.

研究室の管理者が,研究室内でどれだけコミュニケーションが取れているか測定しようとしたとする.


\section{おわりに}
\subsection{まとめ}
研究目的
動作検証では時空間フェンシングが適切に行えているか,実際のユースケースを想定して適切にセンシングできているか検証した.結果何がわかった.
\subsection{今後の課題}
今後の課題として,今回実装に至らなかった時空間フェンシングに基づくクラウドセンシングプラットフォームにおけるモバイルアプリケーションに必要な機能の実装が挙げられる.

時空間フェンシングにGPSを使用しているのでGPSの精度が落ちる屋内で動作が不安定になる点.

% \bibliography{reference}
% \bibliographystyle{junsrt}

% \begin{thebibliography}{9}
%     \bibitem{ura}西村 他.スマートフォンを活用した屋内環境における混雑センシング,情報処理学会論文誌,Vol.55,No.12,pp.2511-2523,(2014).
%     \bibitem{iwa}岩井 他.コミュニティによるスマートフォンを利用した騒音センシングシステムの構築,情報科学技術フォーラム講演論文集,Vol.10,No.4,pp.293-294,(2011).
%     \bibitem{tex}松田 他. 多様なユースケースに対応可能なユーザ参加型モバイルセンシング基盤の実装と評価,マルチメディア,分散,協調とモバイル(DICOMO)シンポジウム論文集,Vol.2016,pp.1042-1050,(2016).
% \end{thebibliography}

\end{document}


% 図を参照するときの番号を自動で入れるには \ref{ラベル名} と書く
% もし,下の図を参照するなら \ref{fig:1} となる.
% 例: システム概要図を図\ref{fig:1}に示す.

% \begin{figure}[H]
%     \centering
%     \includegraphics[width=80mm]{fig1.png}
%     \caption{ラヴラスの流れ}
%     \label{fig:1}
% \end{figure}
% \vspace{-2zh}


%%% Local Variables: 
%%% mode: japanese-latex
%%% TeX-master: t
%%% End: 
